% This is a template following IEEE style
\documentclass[conference]{IEEEtran}

\usepackage{cite}
\usepackage{amsmath,amssymb,amsfonts}
\usepackage{algorithmic}
\usepackage{graphicx}
\usepackage{textcomp}
% code listings package
\usepackage{listings}
% these packages render tables
\usepackage{booktabs}
% the package for references
% \usepackage[hidelinks]{hyperref}
\usepackage{listings}

% force columns to be level on last page (for easier printing in journals)
\usepackage{flushend}
% code highlighting. found in the pandoc default

% override tightlist. error with pandoc?
\def\tightlist{}

% start the document, the markdown code takes over from this point
\begin{document}
\title{The Effect of Cache Memory on Matrix Operations}

\newcommand{\AuthorName}{Christian Kauten}
\newcommand{\AuthorEmail}{jck0022@auburn.edu}

\newcommand{\Deptartment}{Software Engineering and Computer Science}
\newcommand{\University}{Auburn 
University}
\newcommand{\Location}{Auburn, AL, USA}

\author{
  \IEEEauthorblockN{\AuthorName}
  \IEEEauthorblockA{\textit{\Deptartment} \\
  \textit{\University}\\
  \Location \\
  \AuthorEmail}
}

\maketitle

\begin{abstract}

Modern computers use clever tricks to increase performance without incurring
massive financial costs. One such performance improvement is seen in hierarchical memory
where a series of progressively slower memory devices work together to reduce
the latency induced by memory operations. Although these systems work well when
used effectively, naive code can break the system, resulting in bad performance.
These issues become highly apparent when looping over large matrices.
Depending on the architecture of the cache, the ordering of the matrix in
memory, and the code itself, initializing and performing actions on matrices
could take a large amount of time. This paper explores the effects of cache
on matrix operations using C-code. The results show that programmers need pay
close attention to the design of their code to ensure proper utilization of
the underlying hardware through novel means.

\end{abstract}

\begin{IEEEkeywords}
Cache, Memory, Matrix, Transpose
\end{IEEEkeywords}

\section{Introduction}\label{introduction}

\subsection{Hierarchical Memory}\label{hierarchical-memory}

\emph{Hierarchical memory} utilizes a sequence of memory devices and
heuristics to improve performance of memory operations. This sequence
exists to balance performance and cost, allowing machines to take
advantage of more expensive, but faster, memory devices. The machine can
then leverage these devices for recurring or frequently used data to
reduce latency.

\subsubsection{Locality}\label{locality}

In order to best utilize this hierarchy of devices, engineers employ two
similar, but distinct heuristics based on \emph{locality}.
\emph{Temporal locality} states that data used recently is likely to be
used again soon. Without temporal locality, cache hits would rarely
occur as new data from memory would always be needed. \emph{Spatial
locality} is the principle that data near recently used data is likely
to be needed soon. This principle drives engineers to copy surrounding
data to cache when a cache miss occurs for a memory address.

\section{Methodology}\label{methodology}

\subsection{Matrices}\label{matrices}

Matrices in C are stored in memory in row-major order. As such, they
flatten out in memory as a sequential vector of rows. The most efficient
way of \emph{iteratively} traversing this matrix is sequentially
iterating over each cell in each row in the vector. This is know as
\emph{row-wise} traversal. By properly utilizing spatial locality, it
performs better than \emph{column-wise} traversal. With large matrices,
column-wise traversal will generate more cache misses as the algorithm
wont portray the same degree of spatial locality. This is because each
cell in a different row is a full matrix length of data types away from
that cell in memory.

\subsection{Transpose}\label{transpose}

The transpose operator, \(M^T\), flips each value in a 2D matrix \(M\)
by its \(i\) and \(j\) values.

\section{Conclusions}\label{conclusions}
\end{document}
